\section*{Preface}
\addcontentsline{toc}{section}{Preface}

Good to see you, physics learner!

This document is written by learners like you, and it serves a very specific purpose---to help you answer
a type of exam question in CIE. I'm talking about things like:

{%
  \sffamily%
  \begin{itemize}[itemsep=0pt]
    \item Define \textit{elastic potential energy}
    \item State the principle of \textit{conservation of momentum}
    \item State the difference between a \textit{stationary wave} and a \textit{progressive wave}\ldots
    \item Explain the origin of \textit{upthrust} for a body in liquid
    \item Explain the part played by \textit{diffraction} and \textit{interference} in the production of
    the \textit{first order maximum} by the \textit{diffraction grating}.
    \item Distinguish between an \textit{\ce{\alpha}-particle} and a \textit{\ce{\beta^+}-particle}.
  \end{itemize}
}

You get the idea. This type of question, where it asks to \textit{state} or \textit{explain} something
sometimes can turn out to be pretty hard, even if you do have an sound understanding of the concepts
involved and nailed the calculation part. Some people hate these questions a lot. However, it doesn't has
to be this hard.

This document aims to help you get better at answering those questions by showing you how examiner wants
you to answer them, and also showing you what to avoid doing. This is \textsc{not} a list for you to memorize,
instead, you should seek understanding of the logic/reasoning/key ideas behind the answers shown here, which is \textbf{often}
an exact copy from the relevant Mark Scheme, and think about how you could write better answers--such as making
your answer more complete or concise, learning new ways and perspectives to describe/explain a things you understand,
or correcting inaccuracies in your knowledge.

Answers here follows the same style as CIE Mark Scheme, only edited/rephrased to make the meaning clearer. A \point{$n$}
denotes that the expression before \point{$n$} could gain $n$ marks. Some common errors mentioned in examiner reports
are listed here with a \NOT prefix. This indicates that the responses could not gain full mark, and often could
gain no mark at all.

It is worth reminding again that I recommend \textbf{against} anyone memorizing this or any other pre-written `answers'.
It is a terrible waste of time and effort and does little to improve your knowledge of physics.

Happy learning and best wishes for your CIE!

\null\hfill Mao Wtm\\*
\null\hfill January 27, 2018
\clearpage

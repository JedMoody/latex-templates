\defcatagory{Particle and Nuclear Physics}

\begin{question}%
  \qtitle{Distinguish between an \term[a particle]{\ce{\alpha} particle} and a \refterm[beta+ particle]{\ce{\beta^+} particle}.}\qpoint{3}

  Any \point{3} from:

  \begin{itemize}
    \item \ce{\alpha} is 2 \refterm{proton}s and 2 \refterm{neutron}s; \ce{\beta^+} is \refterm{positron}.
    \item \ce{\alpha} has \refterm{charge} $+2e$; \ce{\beta^+} has \refterm{charge} $+e$.
    \item \ce{\alpha} has mass $4u$; \ce{\beta^+} has mass $\frac{1}{2000}u$.
    \item \ce{\alpha} made up of \refterm{hadron}s; \ce{\beta^+} made up of a \refterm{lepton}.
  \end{itemize}
\end{question}

\begin{question}%
  \qtitle{Similarity and difference between a \term[beta+ particle]{\ce{\beta^+} particle} and a \term[beta- particle]{\ce{\beta^-} particle}\ldots}

  Similarity: same (rest) mass, equal magnitude of \refterm{charge}.\\*
  Difference: opposite sign of \refterm{charge}, one is matter / electron and one is \refterm{antimatter} / antielectron / \refterm{positron}.
\end{question}

\begin{question}%
  \qtitle{State the name of the force responsible for \ce{\beta} decay.}\qpoint{1}

  \reftermnoindex[weak nuclear force]{weak (nuclear force/interaction)}\index{weak nuclear force}~\hfill\point{1} \\*
  \NOT simply `nuclear force'
\end{question}

\begin{question}%
  \qtitle{State the quantities that are conserved in a nuclear reaction.}

  Any \point{$n$} from:

  \begin{itemize}
    \item \refterm{mass-energy} \NOT separately `mass' or `energy' or `mass \textit{and} energy'
    \item \refterm{momentum}
    \item \refterm{proton number}
    \item \refterm{nucleon number}
    \item \refterm{charge}
  \end{itemize}

\end{question}

\begin{question}%
  \qtitle{State the names of the \refterm{lepton}s produced in each of the decay processes:}

  \begin{itemize}
    \item \ce{\beta^-} decay: electron and (electron) antineutrino
    \item \ce{\beta^+} decay: positron and (electron) neutrino
  \end{itemize}
\end{question}

\begin{question}%
  \qtitle{State the name of the class (group) to which each of the following belongs\ldots}

  \begin{itemize}
    \item [] \refterm{electron} / \refterm[beta- particle]{\ce{\beta} particle} / neutrino:

      \refterm{lepton}s

    \item [] \refterm{neutron} / \refterm{proton}:

      \refterm{hadron}s \OR \refterm{baryon}s
  \end{itemize}

  \textbf{Please} avoid mis-spelling.
\end{question}

\begin{question}%
  \qtitle{Explain why the sum of the kinetic energies of the carbon-13 nucleus and the \refterm[beta+ particle]{\ce{\beta^+} particle} cannot
    be equal to the total energy released by the decay process \mbox{\ce{X -> {}^{13}_{6} C + \beta^+}}.}

  a (electron) neutrino/\ce{\mathcal{V}_{(e)}} is also produced (and thus has energy)~\hfill\point{1}
\end{question}

\begin{question}%
  \qtitle{State the composition of the proton and of the neutron in terms of \refterm{quark}s.}\qpoint{1}

  \begin{itemize}
    \item proton: up up down (no strange) / $u$ $u$ $d$
    \item neutron: up down down (no strange) / $u$ $d$ $d$
  \end{itemize}
\end{question}

\begin{question}%
  \qtitle{Give one example of \ldots}

  \begin{itemize}
    \item [(i)] \refterm{hadron}: neutron \OR proton
    \item [(ii)] \refterm{lepton}: electron \OR (electron) neutrino
  \end{itemize}
\end{question}

\begin{question}%
  \qtitle{State one difference between a \term{hadron} and a \term{lepton}}\qpoint{1}

  hadrons are not fundamental particle / leptons are fundamental particle \\*
  \OR hadron made of \refterm{quark}s/lepton not made of \refterm{quark}s \\*
  \OR \reftermnoindex[strong nuclear force]{strong force/interaction}\index{strong nuclear force} acts on hadrons/does not act on leptons~\hfill\point{1}\\*
  \NOT comparing mass between \refterm{proton} and \refterm{electron}. \\*
  \NOT `only leptons experience the \reftermnoindex[weak nuclear force]{weak force}\index{weak nuclear force}'
\end{question}

\begin{question}%
  \qtitle{State what may be inferred from the following results in the \refterm[a particle scattering experiment]{\ce{\alpha} particle scattering experiment}.}

  \begin{itemize}
    \item The vast majority of \refterm[a particle]{\ce{\alpha} particle}s pass straight through the metal foil or are deviated by small angles.\qpoint{1}

    \textbf{most} of the \refterm{atom} is empty space \\*
    \OR the nucleus (volume) is (very) small \textbf{compared to the atom}~\hfill\point{1}

    \item A very small minority of \refterm[a particle]{\ce{\alpha} particle}s are scattered through angles greater than \SI{90}{\degree}.\qpoint{2}

    nucleus is (positively) charged~\hfill\point{1}\\*
    the mass/charge is concentrated / the majority of mass/charge in (very small) \refterm{nucleus} / small region/volume/core~\hfill\point{1}

  \end{itemize}

  When asked to state the results, avoid expressions such as `some', `a lot/few' or `many' particles---use `vast majority' or `vast majority'.
\end{question}

\begin{figure}[h]
  \figureruletop

  \caption{Two parallel, vertical metal plates in a vacuum are connected to a power supply, and a radioactive source emitting \refterm[a particle]{\ce{\alpha} particle}
            is placed below the plates. This is for question~\ref{q:a-experiment}~and~\ref{q:a-experiment-2}.}
  \label{fig:a-experiment}

  \centering\begin{tikzpicture}
    \path[fill=black] (0.95, 0) rectangle (1.05, 5) node [below=1, midway, shape=circle, fill=black, inner sep=0.2em] (plateA) {};
    \path[fill=black] (3.95, 0) rectangle (4.05, 5) node [below=1, midway, shape=circle, fill=black, inner sep=0.2em] (plateB) {};
    \path[draw=black, thick] (plateA.center) -- ++(-2, 0) -- ++(0, -4) -- ++(3, 0) node[pos=1, anchor=center, shape=circle, draw=black, fill=white, inner sep=0.2em] (pos) {};
    \path[draw=black, thick] ($(pos.center) + (1, 0)$) -- ++(3, 0) node[pos=0, anchor=center, shape=circle, draw=black, fill=white, inner sep=0.2em] (neg) {} -- ++(0, 4) -- (plateB.center);
    \node[anchor=base] at ($(pos.center) + (0, -0.5)$) {$+$};
    \node[anchor=base] at ($(neg.center) + (0, -0.5)$) {$-$};

    \path (neg.center) -- (pos.center) node[midway, anchor=center] (supplycenter) {};
    \path[fill=black] ($(supplycenter.center) + (-0.5, 1)$) rectangle ($(supplycenter.center) + (0.5, 2)$) node[midway, anchor=center] (source) {};

    \path[draw, thick] (source.center) -- (4.5, 3) node[pos=1, anchor=south west, align=left] {radioactive source};

    \path[draw=white, line width=0.3em] ($(source.center) + (0, 0.25) + (0, -0.1em)$) -- ++(0, 3);
    \path[draw=black, -{Stealth}, line width=0.1em] ($(source.center) + (0, 0.25)$) -- ++(0, 3) node[pos=1, above=0.5em] {\refterm[a particle]{\ce{\alpha} particle}s};
  \end{tikzpicture}

  \figureruletbottom
\end{figure}

\begin{question}%
  \label{q:a-experiment}%
  \qtitle{Explain why the metal plates are placed in a \refterm{vacuum} in figure~\ref{fig:a-experiment}.}\qpoint{1}

  range of \ce{\alpha} particle is only few \SI{}{cm} in air \\*
  \OR loss of energy of the \ce{\alpha} particles due to collision with air molecules/ionisation of the air molecules~\hfill\point{1}
\end{question}

\begin{question}%
  \label{q:a-experiment-2}%
  \qtitle{The \refterm[a particle]{\ce{\alpha} particle} source in figure~\ref{fig:a-experiment} is replaced by a \refterm[beta- particle]{\ce{\beta} particle} source. By reference to the
    properties of \refterm[a particle]{\ce{\alpha} particle} and \refterm[beta- particle]{\ce{\beta} particle}, suggest three possible differences in the
    deflection observed with \refterm[beta- particle]{\ce{\beta} particle}.}\qpoint{3}

  \ce{\beta} have opposite \refterm{charge} to \ce{\alpha} therefore deflection in opposite direction~\hfill\point{1}\\*
  \ce{\beta} has a range of velocities/energies hence a number of different deflections would be seen~\hfill\point{1}\\*
  \ce{\beta} have less \refterm{mass} or $\frac{\text{\refterm{charge}}}{\text{\refterm{mass}}}$ is larger hence deflection is greater \\*
  \OR \ce{\beta} with (very) high speed (may) have less deflection~\hfill\point{1}

  There must be references to properties, such as `opposite \refterm{charge}'.
\end{question}

\begin{question}%
  \qtitle{State the \refterm{constituent particles} of the \textit{<some element>} nucleus}\qpoint{1}

  $x$ \refterm{proton}s and $y$ \refterm{neutron}s~\hfill\point{1}\\*
  \NOT $x$ electrons.
\end{question}

\begin{question}%
  \qtitle{State the \refterm{constituent particles} of \refterm[a particle]{\ce{\alpha} particle}}\qpoint{1}

  2 \refterm{proton}s and 2 \refterm{neutron}s~\hfill\point{1}\\*
  \NOT \ce{^4_2 He} / Helium / Helium nucleus
\end{question}

\begin{question}%
  \qtitle{Explain, using the law of \term{mass-energy conservation}, how \refterm{energy} is released in a \refterm{nuclear reaction}}\qpoint{2}

  (total) \refterm{mass} on left-hand side (of equation)/reactants is greater than (total) mass on right-hand side (of equation)/products~\hfill\point{1}\\*
  difference in mass is (converted to) \refterm{energy}.\hfill\point{1}
\end{question}

\begin{question}%
  \qtitle{Explain the meaning of \term{spontaneous radioactive decay}}\qpoint{1}

  (rate of decay) not affected by any external factors or changes in temperature and pressure etc.~\hfill\point{1}\\*
  \NOT decay occurred randomly / naturally
\end{question}
